\usepackage{listings}

\lstset{literate=
  {á}{{\'a}}1 {é}{{\'e}}1 {í}{{\'i}}1 {ó}{{\'o}}1 {ú}{{\'u}}1
  {Á}{{\'A}}1 {É}{{\'E}}1 {Í}{{\'I}}1 {Ó}{{\'O}}1 {Ú}{{\'U}}1
  {à}{{\`a}}1 {è}{{\`e}}1 {ì}{{\`i}}1 {ò}{{\`o}}1 {ù}{{\`u}}1
  {À}{{\`A}}1 {È}{{\'E}}1 {Ì}{{\`I}}1 {Ò}{{\`O}}1 {Ù}{{\`U}}1
  {ä}{{\"a}}1 {ë}{{\"e}}1 {ï}{{\"i}}1 {ö}{{\"o}}1 {ü}{{\"u}}1
  {Ä}{{\"A}}1 {Ë}{{\"E}}1 {Ï}{{\"I}}1 {Ö}{{\"O}}1 {Ü}{{\"U}}1
  {â}{{\^a}}1 {ê}{{\^e}}1 {î}{{\^i}}1 {ô}{{\^o}}1 {û}{{\^u}}1
  {Â}{{\^A}}1 {Ê}{{\^E}}1 {Î}{{\^I}}1 {Ô}{{\^O}}1 {Û}{{\^U}}1
  {Ã}{{\~A}}1 {ã}{{\~a}}1 {Õ}{{\~O}}1 {õ}{{\~o}}1
  {œ}{{\oe}}1 {Œ}{{\OE}}1 {æ}{{\ae}}1 {Æ}{{\AE}}1 {ß}{{\ss}}1
  {ű}{{\H{u}}}1 {Ű}{{\H{U}}}1 {ő}{{\H{o}}}1 {Ő}{{\H{O}}}1
  {ç}{{\c c}}1 {Ç}{{\c C}}1 {ø}{{\o}}1 {å}{{\r a}}1 {Å}{{\r A}}1
  {€}{{\euro}}1 {£}{{\pounds}}1 {«}{{\guillemotleft}}1
  {»}{{\guillemotright}}1 {ñ}{{\~n}}1 {Ñ}{{\~N}}1 {¿}{{?`}}1
}

\lstdefinestyle{C}{
	backgroundcolor=\color{white},   % choose the background color; you must add \usepackage{color} or \usepackage{xcolor}; should come as last argument
	basicstyle=\footnotesize\ttfamily,        % the size of the fonts that are used for the code
	belowcaptionskip=1\baselineskip,
	breakatwhitespace=false,         % sets if automatic breaks should only happen at whitespace
	breaklines=true,                 % sets automatic line breaking
	captionpos=b,                    % sets the caption-position to bottom
	commentstyle=\color{green!40!black},,    % comment style
	deletekeywords={...},            % if you want to delete keywords from the given language
	escapeinside={\%*}{*)},          % if you want to add LaTeX within your code
	extendedchars=true,              % lets you use non-ASCII characters; for 8-bits encodings only, does not work with UTF-8
	firstnumber=1,                % start line enumeration with line 1000
	frame=leftline,	                   
	identifierstyle=\color{blue!50!black},
	keepspaces=true,                 % keeps spaces in text, useful for keeping indentation of code (possibly needs columns=flexible)
	keywordstyle=\bfseries\color{violet},       % keyword style
	language=C,                 % the language of the code
	morekeywords={*,...},            % if you want to add more keywords to the set
	numbers=left,                    % where to put the line-numbers; possible values are (none, left, right)
	numbersep=10pt,                   % how far the line-numbers are from the code
	numberstyle=\tiny\color{gray}, % the style that is used for the line-numbers
	rulecolor=\color{black},         % if not set, the frame-color may be changed on line-breaks within not-black text (e.g. comments (green here))
	showspaces=false,                % show spaces everywhere adding particular underscores; it overrides 'showstringspaces'
	showstringspaces=false,          % underline spaces within strings only
	showtabs=false,                  % show tabs within strings adding particular underscores
	stepnumber=1,                    % the step between two line-numbers. If it's 1, each line will be numbered
	stringstyle=\color{orange},
	tabsize=2,	                   % sets default tabsize to 2 spaces
	title=\lstname,
	xleftmargin=1.5cm,
}

\definecolor{mygreen}{rgb}{0.1,0.4,0.1}
\lstdefinestyle{console}{
	basicstyle=\footnotesize\ttfamily\color{black},
	commentstyle=\color{mygreen},
	keywordstyle=\color{blue},
    numberstyle=\tiny\color{gray},
    stringstyle=\color{Purple},
    breakatwhitespace=false,         
    breaklines=true,                 
    captionpos=b,                    
    keepspaces=true,                 
    numbers=none,                    
    numbersep=5pt,                  
    showspaces=false,                
    showstringspaces=false,
    showtabs=false,                  
    tabsize=2
}



\lstdefinestyle{bash} {
	basicstyle=\footnotesize\ttfamily\color{black},
	language=bash,
	commentstyle=\color{mygreen},
	showspaces=false,
	breaklines=true,  
	showstringspaces=false
}


\lstdefinestyle{Python}{
	keywords=[1]{range, reversed, for, in, import, from, self},
	keywords=[2]{list, tuple, int, str, bytearray, bool},
	keywords=[3]{class, def},
	keywordstyle=[1]\bfseries\color{Violet},
	keywordstyle=[2]\bfseries\color{ForestGreen},
	keywordstyle=[3]\bfseries\color{RoyalBlue},
	backgroundcolor=\color{white},   % choose the background color; you must add \usepackage{color} or \usepackage{xcolor}; should come as last argument
	basicstyle=\footnotesize\ttfamily\color{black},        % the size of the fonts that are used for the code
	belowcaptionskip=1\baselineskip,
	breakatwhitespace=false,         % sets if automatic breaks should only happen at whitespace
	breaklines=true,                 % sets automatic line breaking
	captionpos=b,                    % sets the caption-position to bottom
	deletekeywords={...},            % if you want to delete keywords from the given language
	escapeinside={\%*}{*)},          % if you want to add LaTeX within your code
	extendedchars=true,              % lets you use non-ASCII characters; for 8-bits encodings only, does not work with UTF-8
	firstnumber=1,                % start line enumeration with line 1000
	frame=leftline,	                   
	%identifierstyle=\color{blue!50!black},
	keepspaces=true,                 % keeps spaces in text, useful for keeping indentation of code (possibly needs columns=flexible)
	language=Python,
	%morekeywords={*,...},            % if you want to add more keywords to the set
	numbers=left,                    % where to put the line-numbers; possible values are (none, left, right)
	numbersep=10pt,                   % how far the line-numbers are from the code
	numberstyle=\tiny\color{gray}, % the style that is used for the line-numbers
	rulecolor=\color{black},         % if not set, the frame-color may be changed on line-breaks within not-black text (e.g. comments (green here))
	morecomment=[l][\color{green!40!black}]{\#},
	morestring=[s]{"""}{"""},
	morestring=[s]{"}{"},
	morestring=[s]{'}{'},
	showspaces=false,                % show spaces everywhere adding particular underscores; it overrides 'showstringspaces'
	showstringspaces=false,          % underline spaces within strings only
	showtabs=false,                  % show tabs within strings adding particular underscores
	stepnumber=1,                    % the step between two line-numbers. If it's 1, each line will be numbered
	stringstyle=\color{orange},
	tabsize=2,	                   % sets default tabsize to 2 spaces
	title=\lstname,
	xleftmargin=1.5cm,
}
