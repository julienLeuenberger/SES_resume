\section{Hardening Linux}
\subsection{Contôler l'intégrité d'un package, d'un programme}
En premier télécharger le programme foo et sa signature ensuite il faut vérifier l'intégrité avec
\begin{Verbatim}[breaklines=true, breakanywhere=true]
gpg --verify foo.tar.gz.asc
\end{Verbatim}
Il nous répond qu'il peut pas en donnant une RSA key: par ex. 23948UT5. Donc il faut importer cette clé publique:
\begin{Verbatim}[breaklines=true, breakanywhere=true]
gpg --keyserver keyserver.ubuntu.com --search-keys 23948UT5
\end{Verbatim}
Et relancer la vérification et c'est bon.
\subsection{Configurer un nouveau package, programme}
Détarer le package et afficher les options de compilationss:
\begin{Verbatim}[breaklines=true, breakanywhere=true]
tar xvzf foo.tar.gz
cd foo
./configure --help
\end{Verbatim}
Si il n'existe pas encore dans buildroot (donc une version inférieure n'a jamais été installé) on le place dans \verb!/buildroot/packages!.
Et on va faire les choix de versions dans le \verb!foo.mk! et ensuite on le configure avec \verb!make menuconfig! le choix de config sont décris dans \verb!Config.in!

\subsection{Cross-compiler un programme}

