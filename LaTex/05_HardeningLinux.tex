\section{Hardening Linux}
\subsection{Contôler l'intégrité d'un package, d'un programme}
En premier télécharger le programme foo et sa signature ensuite il faut vérifier l'intégrité avec
\begin{lstlisting}[style=bash]
gpg --verify foo.tar.gz.asc
\end{lstlisting}
Il nous répond qu'il peut pas en donnant une RSA key: par ex. 23948UT5. Donc il faut importer cette clé publique:
\begin{lstlisting}[style=bash]
gpg --keyserver keyserver.ubuntu.com --search-keys 23948UT5
\end{lstlisting}
Et relancer la vérification et c'est bon.
\subsection{Configurer un nouveau package, programme}
Détarer le package et afficher les options de compilationss:
\begin{lstlisting}[style=bash]
tar xvzf foo.tar.gz
cd foo
./configure --help
\end{lstlisting}
Si il n'existe pas encore dans buildroot (donc une version inférieure n'a jamais été installé) on le place dans \verb!/buildroot/packages!.
Et on va faire les choix de versions dans le \verb!foo.mk! et ensuite on le configure avec \verb!make menuconfig! le choix de config sont décris dans \verb!Config.in!

\subsection{Cross-compiler un programme}
Ajout de host et prefix
\begin{lstlisting}[style=console]
./configure --host=aarch64-none-linux-gnu --prefix=/root/install
make 
make install
\end{lstlisting}

\subsection{Contrôler les services, les ports ouverts}
Les services sont lancés au démarrage si leur Startup scripts se trouve dans le \verb!/etc/init.d!
Le script \verb!rcS! commence les autres scripts (S01..., S02..., not rcK) dans l'ordre alphabétique. On essaie d'avoir le moins de services possibles. On est obligé de créer un script pour démarrer automatiquement un nouveau service. Si on supprime un S0.... le service ne sera pas débuté au prochain startup. Pour voir les services en en cours: \verb!ps -ale!, le tree: \verb!ps -axjf!, leurs privilèges: \verb!ps aux!, montrer les fichiers ouverts et leurs privilèges: \verb!lsof | grep sshd!. Pour contrôler les ports ouverts, \verb!netstat -atun! montre les tcp and udp, \verb!lso | grep IP! montre les fichiers ouverts et \verb!nmap! (scan all tcp ports for the host 192.168.0.11) et (// scan all udp ports for the host 192.168.0.11)
\begin{lstlisting}[style=bash]
nmap -Pn -p 1-65535 192.168.0.11 
nmap -sU -Pn -p 1-65535 192.168.0.11 
\end{lstlisting}

\subsection{Contrôler les files systems}
Leur structure se trouve dans \verb!/etc/stab!

\subsection{Contrôler les permission des fichiers, répertoires}
Les users normaux devraient pas avoir accès au syslog files \verb!/var/log!. Aucun fichier ou repertoires writable devrait exister, à part pour les temporary directories \verb!/tmp!, \verb!/usr/tmp! et \verb!/var/tmp! (\verb!ls /! ou \verb!find /etc -type d -perm -0002 -print!) 
Les fichiers dans \verb!/etc! must not have group and other bits writable (\verb!find /etc -type f -perm -0002 -print! ou \verb!find /etc -type f -perm -0020 -print!). SUID ou SGID binaries devraient être désactivés: (\verb!find / -perm -4000 -print! ou \verb!find / -perm -2000 -print!). Faudrait forcer \verb!/tmp! à être cleaned durant le shutdown et encore mieux si possible: on met \verb!/tmp! sur un disque en ram

\subsection{Sécuriser le réseau}
De base, trop d'options sont actvées et donnent trop d'informations au hacker. \\
Disable IPV6
\begin{lstlisting}[style=bash]
echo 1 > /proc/sys/net/ipv6/conf/all/disable_ipv6
echo 1 > /proc/sys/net/ipv6/conf/default/disable_ipv6
Or sysctl -w /net/ipv6/conf/default/disable_ipv6=1
\end{lstlisting}
IP source routing must be disabled
\begin{lstlisting}[style=bash]
echo 0 > cat /proc/sys/net/ipv4/conf/all/accept_source_route
Or sysctl -w net/ipv4/conf/all/accept_source_route=0
\end{lstlisting}
Forwarding (Routing) of normal and multicast packets should also be deactivated unless 
expressively needed
\begin{lstlisting}[style=bash]
echo 0 > /proc/sys/net/ipv4/conf/all/forwarding
echo 0 > /proc/sys/net/ipv6/conf/all/forwarding
echo 0 > /proc/sys/net/ipv4/conf/all/mc_forwarding
echo 0 > /proc/sys/net/ipv6/conf/all/mc_forwarding
Or sysctl -w net/ipv4/conf/all/forwarding=0
sysctl -w net/ipv6/conf/all/forwarding=0
sysctl -w net/ipv4/conf/all/mc_forwarding=0
sysctl -w net/ipv6/conf/all/mc_forwarding=0
\end{lstlisting}
Block ICMP redirect messages
\begin{lstlisting}[style=bash]
echo 0 > /proc/sys/net/ipv4/conf/all/accept_redirects
echo 0 > /proc/sys/net/ipv6/conf/all/accept_redirects
echo 0 > /proc/sys/net/ipv4/conf/all/secure_redirects
echo 0 > /proc/sys/net/ipv4/conf/all/send_redirects
Or sysctl -w net/ipv4/conf/all/accept_redirects=0
sysctl -w net/ipv6/conf/all/accept_redirects=0
sysctl -w net/ipv4/conf/all/secure_redirects=0
sysctl -w net/ipv4/conf/all/send_redirects =0
\end{lstlisting}
Enable source route verification in order to prevent spoofing
\begin{lstlisting}[style=bash]
echo 1 > /proc/sys/net/ipv4/conf/default/rp_filter
Or sysctl -w net/ipv4/conf/default/rp_filter=0
\end{lstlisting}
Log all malformed packed and ignore icmp bogus ones
\begin{lstlisting}[style=bash]
echo 1 > /proc/sys/net/ipv4/conf/all/log_martians
echo 1 > /proc/sys/net/ipv4/icmp_ignore_bogus_error_responses
Or sysctl -w net/ipv4/conf/all/log_martians=1
sysctl -w net/ipv4/icmp_ignore_bogus_error_responses=1
\end{lstlisting}
Disable ICMP echo and timestamp responses sent via broadcast or multicast
\begin{lstlisting}[style=bash]
echo 1 > /proc/sys/net/ipv4/icmp_echo_ignore_broadcasts
Or sysctl -w net/ipv4/icmp_echo_ignore_broadcasts=1
\end{lstlisting}
Increase resilience under heavy TCP load by increasing backlog buffer and by enabling 
syn cookies
\begin{lstlisting}[style=bash]
echo 4096 > /proc/sys/net/ipv4/tcp_max_syn_backlog
echo 1 > /proc/sys/net/ipv4/tcp_syncookies
Or sysctl -w net/ipv4/tcp_max_syn_backlog=4096
sysctl -w net/ipv4/tcp_syncookies=1
\end{lstlisting}
la commande \verb!systctl -p! montre le contenu \verb!/etc/sysctl.conf!

\subsection{Contrôler-sécuriser les comptes utilisateurs}
Ils sont affichés dans \verb!/etc/shadow!. Le umask(user file creation right) doit être 0027 dans \verb!cat /etc/profile.d/umask.sh! et ça met 0 dans les others.
On peut aussi ajouter un message disant que c'est interdit avant le login dans le fichier \verb!/etc/issue!

\subsection{Limiter le login root}
Le root devrait être uniquement possible par console : \verb!echo "ttyS0" > /etc/securetty!. Seulement le root peut accéder au dir root: \verb!chmod 700/root!. On utiliser \verb!sudo! pour prendre les droits de roots. Le PATH du root ne doit pas contenir le current dir ou des writable dir: \verb!PATH not equal .:/tmp:/var/tmp: etc! 
\subsection{Sécuriser le noyau}
ASLR (comme en dessus) +write protect kernel make linux-config
\begin{lstlisting}[style=bash]
Kernel Feature => [*] Apply r/o permissions of VM areas also to their linear aliases
General Setup => [*] Set loadable kernel module data as NX and text as RO
\end{lstlisting}
Strip-assembler comme plus haut, -fstack-protector, restrict unprivilegiend access to kernel syslog

\subsection{Sécuriser une application}
Ajouter les \verb!CFLAGS! et \verb!LDFLAGS! suivant
\begin{lstlisting}[style=bash]
CFLAGS="-fPIE -fstack-protector-all -D\_FORTIFY\_SOURCE=2 -ftrapv"
LDFLAGS="-Wl,-z,now,-z,relro -z,noexecstack, -pie"
\end{lstlisting}
\begin{itemize}
\item PIE: position independant executable pour avoir ROP
\item -fstack-protector-all: ajout de canary random
\item D\_FORTIFY\_SOURCE=2: overflow check for memcpy, ...
\item ftrapv: integer (only) overflow activate a catcheable signal SIGABRT
\item -Wl,-z,now,-z,relro: tous les symbols dynamiques sont résolut durant le début du porgramme et ça empêche des overwrite attacks
\end{itemize}

\subsection{Contrôler le démarrage de Linux}
Une solution, dans le but de vérifier qu'on boot sur le bon kernel, consisterait à stocker le hash du kernel sur la partition de boot et avant de booter linux faire le hash de kernel depuis uboot et comparer avec la valeur en partition. 
\subsection{Méthodologie OSSTMM simplifiée}
\begin{enumerate}
\item Access: compter le nombre d'accès possible, le nombre de port TCP/UDP ouverts (plus haut) + \verb!netsat -atunp!
\item Authentification: compter le nombre d'accès (ssh, console, ...) avec username et password
\item Confidentiality: compter le nombre de connexion où les données sont cryptées (ssh est mieux telnet)
\item Vulnerability, Weekness, Concern: check si un program a des vulnérabilité sur internet, tester les passwords avec les hashcat, .. et tester la robustesse des algorithme de cryptographie
\item Exposure: Ne pas donner la version de openssh sur nmap, c'est débile
\end{enumerate}
