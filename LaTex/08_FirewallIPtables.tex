\section{Firewall IPtables}
\subsection{Principes de Netfilter et iptables}
il faut configurer le kernel avec make linux-menuconfig. Iptables est un programme qui sert à configuer Netfilter qui est dans le noyau linux. La frame qui entre passe à tavers la stack du protocol et le netfilter sera appelé. Les principales features de iptables sont stateless et stateful packet filtering, network adress translation, ... On peut créer des parefeu, faire un NAT si on a pas assez d'adresse IP.
\subsection{Chain-tables}
Quand un paquet arrive, il traverse différentes chains qui contiennent des tables. Les chains: INPUT (le paquet vient pour le host local), OUTPUT (le paquet est sent by the local host), FORWARD (vient eth0 et va à eth1), PREROUTING et POSTROUTING (used by NAT). Les tables: Fiter table agit sur INPUT, FORWARD et OUTPUT, NAT table (Network Address Translation au moment d'une nouvelle connexion) agit sur PREROUTING, OUTPUT ET POSTROUTING et Mangle table (permet de modifier des paquets) agit sur TOUS les chains.
\subsection{Différences entre stateless et stateful}
Stateless: peut importe l'état (est-ce qu'il a déjà eu une connexion, ...), on refuse on accepte.

\subsection{Configuration d'un stateless et d'un stateful}
Stateless
\begin{Verbatim}[breaklines=true, breakanywhere=true]
# list all existing rules in the filter table
iptables -L -n -v -t filter
# flush tables
iptables -F -t filter
# Create new chain
iptables -N dynamic
#This command ignores everything coming from 200.200.200.1, -A = append
iptables -A INPUT -s 200.200.200.1 -j DROP
# delete previous rule
iptables -D INPUT -s 200.200.200.1 -j DROP
# log et drop tout le icmp (ping)
iptables -A INPUT -p icmp -j LOG
iptables -A INPUT -p icmp -j DROP
# Accept the ssh service from host 10.0.0.1
iptables -A INPUT -p tcp -s 10.0.0.1 -d 0/0 --dport 22 -j ACCEPT
# Accept the ssh service from subnet 10.0.0.x
iptables -A INPUT -p tcp -s 10.0.0.0/24 -d 0/0 --dport 22 -j ACCEPT
# Reject all incoming TCP traffic destined for ports 0 to 1023
iptables -A INPUT -p tcp -s 0/0 -d 0/0 --dport 0:1023 -j REJECT
# Reject all outgoing TCP traffic except the one destined for 10.0.0.1
iptables -A OUTPUT -p tcp -s 0/0 -d ! 10.0.0.1 -j REJECT
# Drop all SYN packets from 10.0.0.1
iptables -A INPUT -p tcp -s 10.0.0.1 -d 0/0 --syn -j DROP
# Change the FORWARD chain to an ACCEPT policy
iptables -P FORWARD ACCEPT
# Block a incoming telnet only for the port ppp0
iptables -A INPUT -p tcp --destination-port telnet -i ppp0 -j DROP
\end{Verbatim}
Statefull, -m connbytes : combien de bytes par paquets en moyenne ont déjà été transférés (pour mettre les download on lower priority).
\begin{Verbatim}[breaklines=true, breakanywhere=true]
# Local traffic interne ok, from inside to outside ok si etabli tout le teste non
iptables -A INPUT -i lo -j ACCEPT
iptables -A OUTPUT -m conntrack --ctstate NEW,ESTABLISHED -j ACCEPT
iptables -A OUTPUT –j DROP
iptables -A INPUT -m conntrack --ctstate ESTABLISHED -j ACCEPT
iptables -A INPUT -p tcp --dport 80 -j ACCEPT
iptables -A INPUT -j DROP
\end{Verbatim}
Interne permis mais que du ssh depuis l'extérieur
\begin{Verbatim}[breaklines=true, breakanywhere=true]
iptables -A INPUT -i lo -j ACCEPT // Local traffics are allowed between applications
iptables -P INPUT DROP
iptables -P OUTPUT DROP
iptables -P FORWARD DROP
iptables -A INPUT -i eth0 -p tcp --dport 22 -m conntrack --ctstate NEW,ESTABLISHED -j ACCEPT
iptables -A OUTPUT -o eth0 -p tcp --sport 22 -m conntrack --ctstate ESTABLISHED -j ACCEPT
\end{Verbatim}
Seulement SSH connexion, et sort que de connexions établies
\begin{Verbatim}[breaklines=true, breakanywhere=true]
iptables -A INPUT -i lo -j ACCEPT // Local traffics are allowed between applications
iptables -P INPUT DROP
iptables -P OUTPUT DROP
iptables -P FORWARD DROP
iptables -A INPUT -i eth0 -p tcp --dport 22 -m conntrack --ctstate NEW,ESTABLISHED -j ACCEPT
iptables -A INPUT -i eth0 -m conntrack --ctstate ESTABLISHED,RELATED -j ACCEPT
iptables -A OUTPUT -o eth0 -p tcp --sport 22 -m conntrack --ctstate ESTABLISHED -j ACCEPT
iptables -A OUTPUT -o eth0 -m conntrack --ctstate NEW,ESTABLISHED,RELATED -j ACCEPT
\end{Verbatim}
Nanopi permet une connexion ssh, sur le port 62200 après authentification, accepter tous les messages de 192.168.0.4
\begin{Verbatim}[breaklines=true, breakanywhere=true]
iptables -A INPUT -i lo -j ACCEPT // Local traffics are allowed between applications
iptables -P INPUT DROP
iptables -P OUTPUT ACCEPT
iptables -P FORWARD DROP
iptables -N dynamic
iptables -N net2fw // New chain: Network to Firewall
iptables -A INPUT -i eth0 -j net2fw
iptables -A net2fw -m conntrack --ctstate NEW -j dynamic
iptables -A net2fw -m conntrack --ctstate RELATED,ESTABLISHED -j ACCEPT
iptables -A net2fw -p tcp -m tcp --dport 62200 -j ACCEPT
iptables -A net2fw -j LOG --log-prefix "net2fw:DROP:" --log-level 6
iptables -A net2fw -j DROP
#After ssh authentication, insert a new rule in the dynamic chain
iptables -A dynamic -s 192.168.0.4 -j ACCEPT
\end{Verbatim}

\subsection{NFQUEUE}
Les cibles NFQUEUE doivent faire en sorte que le paquet aille à l'userspace. Typiquement l'icmp:
\begin{Verbatim}[breaklines=true, breakanywhere=true]
iptables -A INPUT -t filter -p icmp -j NFQUEUE --queue-num=0
\end{Verbatim}




