\section{Kernel}
\subsection{Principaux répertoire du noyau linux}
Les principaux répertoires du kernel sont listés:
\begin{itemize}
\item arch Hardware dependent code
\item block Generic functions for the block devices
\item crypto Cryptographic algo. used in the kernel
\item Documentation Documentation about the kernel
\item drivers All drivers known by the kernel
\item fs All filesystem know by the kernel
\item include kernel include files
\item init Init code (function start\_kernel)
\item ipc Interprocess communication
\item kernel Kernel code, scheduler, mutex, … 
\item lib different libraries used by the kernel
\item mm Memory management
\item net Different protocols, IPv4, IPv6, bluetooth, ...
\item samples Different examples, kobject, kfifo, ...
\item security Encrypted keys, SELinux, ...
\item sound Sound support for Linux kernel
\item virt Kernel-based virtual machine
\end{itemize}
\subsection{Principales méthodes pour sécuriser le noyau}
Pour configurer linux : \verb!make linux-menuconfig! ou \verb!make linux-xconfig!
\begin{itemize}
\item ne pas inclure le Linux kernel \verb!.config! dans le kernel : \verb!Kernel .config support!
\item protéger la stack en activant la détection de buffer overflow : General architecture-dependant options : [*] Stack Protector buffer overflow detection, [*] Strong Stack Protector
\item 
\end{itemize}